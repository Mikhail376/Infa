\documentclass[paper=b5,pagesize]{scrartcl}


% Русский, шрифты
\usepackage{fontspec}
\setmainfont{Times New Roman}
\usepackage{polyglossia}
\setmainlanguage{russian}
\linespread{0.89}

% Формат страницы B5 + поля
\usepackage{geometry}
\geometry{
  paperwidth=176mm,
  paperheight=250mm,
  top=20mm,
  bottom=20mm,
  left=16mm,
  right=13mm,
  columnsep=6mm
}

% Две колонки + перенос формул
\usepackage{multicol}
\usepackage{mathtools} % для переноса формул
\usepackage{array} % для таблиц
\raggedcolumns
\setlength{\parskip}{0.15ex}
\setlength{\premulticols}{0pt}
\setlength{\postmulticols}{0pt}
\setlength{\multicolsep}{1mm}

% Разрешаем перенос длинных формул
\allowdisplaybreaks

% Колонтитулы
\usepackage{scrlayer-scrpage}
\clearpairofpagestyles
\setlength{\headheight}{22pt}
\chead{\large\bfseries Ответы, указания, решения}
\KOMAoptions{headsepline=0.4pt:0.92\textwidth}

% УСТАНОВКА КОЛОНТИТУЛА:
\ifoot*{\large \thepage} 

% TikZ и математика
\usepackage{tikz}
\usepackage{amsmath,amssymb}

\begin{document}


% Устанавливаем начальный номер страницы на 78
\setcounter{page}{78} 

\vspace*{-1mm}

% ТЕКСТ ДВЕ КОЛОНКИ
\begin{multicols}{2}
\small

\textbf{К статье «Объем тел вращения»}

1. a) \(\dfrac{4}{3}\pi R^3\cos\theta\); б) \(2\pi^2 R^2 m\cos\theta\).

2. \(4\pi R^3 (\pi + 4)\).

\textbf{Указание.}
Объем тела, полученного вращением цилиндра, равновелик сумме объемов тел, полученных вращением его проекций на плоскость \(\alpha\) и перпендикулярную ей плоскость, проходящую через \(l\).

\textbf{К статье «Об одном геометрическом месте точек»}

1. Взять \(A_1\), совпадающий с \(A\), \(B_1 = c - B\), \(C_1 = c - C\).

2. Возьмите в качестве \(A,B,C\) центры окружностей; в качестве \(A_1,B_1,C_1\) — по одной из точек пересечения соответственно окружностей с центрами \(B\) и \(C\), \(A\) и \(C\), \(A\) и \(B\) и воспользуйтесь утверждением 3.

3. Доказательство аналогично доказательству необходимости в утверждении 3.

4. Обозначим стороны треугольника через \(a,b,c\). Каждая сторона делится точкой касания на две части, длины которых легко вычисляются. Далее легко проверить равенство (1) в статье.

5. Если бы такая точка нашлась (обозначим её через \(M\)), то прямая \(MN\) была бы перпендикулярна всем сторонам треугольника \(ABC\).

6. Пусть \(BD=x\), \(DC=y\), \(AD=z\), а сторона треугольника \(ABC\) равна \(a\). Если теперь \(A_2,B_2,C_2\) — точки касания соответствующих окружностей со сторонами \(BC\), \(AC\) и \(AB\), то отрезки \(BA_2,A_2C\) и т.д. легко вычисляются (в частности, \(BA_2=\frac{a+x-y}{2}\)).

Обозначим через \(U,V,W\) радиусы окружностей, вписанных в треугольники \(BDC,ADC\) и \(ADB\). Теперь легко проверить справедливость равенства (1) в утверждении 3.

7. Если \(A_1,A_2,A_3,A_4\) образуют выпуклый четырёхугольник, \(D\) — точка пересечения диагоналей, \(A_1D=m\), \(A_2D=p\), \(A_3D=n\), \(A_4D=q\), то для любой точки \(M\) плоскости будет постоянно выражение
\[
\begin{aligned}
&\frac{n}{n+m}(A_1M)^2 - \frac{q}{q+p}(A_2M)^2 \\
&+ \frac{m}{m+n}(A_3M)^2 - \frac{p}{q+p}(A_4M)^2.
\end{aligned}
\]

8. Пусть \(AM_1 : BM_1 : CM_1 = p:q:r\). Тогда геометрическим местом точек \(M\) таких, что \((p^2-q^2)(AM)^2 + (p^2-r^2)(BM)^2 + (q^2-p^2)(CM)^2 = 0\), будет прямая, проходящая через \(M_1,M_2\) и центр описанного около \(ABC\) круга.

\columnbreak

9. Пусть \(O\) — центр окружности, вписанной в \(\triangle ABC\). Поскольку \(CM=CN\), то \(MN\perp OC\), аналогично \(PK\perp BO\). Следовательно, нужно доказать, что перпендикуляры, опущенные из \(N,P,A\) на \(OC,BO,BC\), пересекаются в одной точке. Для этого проверьте выполнение равенства (1).

10. Поскольку площади треугольников \(ABM\) и \(CDM\) не меняются при движении отрезков \(AB\) и \(CD\) соответственно по прямым \(AB\) и \(CD\), то можно передвинуть эти отрезки вдоль соответствующих прямых так, что их концы совпадут с точкой пересечения прямых \(AB\) и \(CD\).

Искомым геометрическим местом точек будет параллелограмм, диагонали которого лежат на прямых \(AB\) и \(CM\).

11. Пусть \(ABCD\) — данный четырехугольник, \(P\) — середина \(AC\), \(Q\) — середина \(BD\), \(O\) — центр окружности. Проверьте, что
\[
\begin{aligned}
S_{\triangle PAB} &+ S_{\triangle PCD} = S_{\triangle QAB} + S_{\triangle QCD} \\
&= S_{\triangle OAB} + S_{\triangle OCD} = \frac12 S_{ABCD}.
\end{aligned}
\]

\textbf{К статье «О графах с цветными рёбрами»}

1. Пусть красное ребро означает, что два делегата могут объясниться на одном языке, а синее — что не могут. Если красного треугольника нет, то по свойству 2 должен быть треугольник с синими сторонами. Но это противоречит условию.

2. Проведённому отрезку поставим в соответствие красное ребро, не проведённому — синее. Докажем, что в полном графе с девятью вершинами, каждая из которых принадлежит четырём красным рёбрам и четырём синим, найдётся треугольник с красными сторонами.

% РИСУНОК
\vspace{2mm}
\begin{flushleft}
% Чтобы ширина не превышала ширину колонки
\begin{minipage}{\linewidth}

% Чтобы рисунок влез в колонку по ширине
\begin{tikzpicture}[scale=0.65]
\draw[thin] (0,0) rectangle (10,6);

\coordinate (A) at (5.5,4.6);
\coordinate (B1) at (1.6,4.0);
\coordinate (B2) at (1.0,3.0);
\coordinate (B3) at (1.0,1.7);
\coordinate (B4) at (1.8,0.9);
\coordinate (C1) at (8.2,3.0);
\coordinate (C2) at (7.8,2.0);
\coordinate (C3) at (5.8,1.2);
\coordinate (C4) at (7.6,3.9);

\foreach \B in {B1,B2,B3,B4}
  \draw[line width=1pt, red!80!black] (A) -- (\B);
\foreach \C in {C1,C2,C3,C4}
  \draw[line width=1pt, blue!70!black] (A) -- (\C);

% СИНИЕ ЛИНИИ МЕЖДУ ТОЧКАМИ B
\draw[line width=1pt, blue!70!black] (B1) -- (B2) -- (B3) -- (B4) -- (B2) -- (B1) -- (B3);

\foreach \P in {A,B1,B2,B3,B4,C1,C2,C3,C4}
  \fill (\P) circle (2pt);

\node[above] at (A) {A};
\node[left] at (B1) {B$_1$};
\node[left] at (B2) {B$_2$};
\node[left] at (B3) {B$_3$};
\node[left] at (B4) {B$_4$};
\node[right] at (C1) {C$_3$};
\node[right] at (C2) {C$_2$};
\node[right] at (C3) {C$_1$};
\node[right] at (C4) {C$_4$};

\end{tikzpicture}

\vspace{1mm}
{\small Рис. 1}

\end{minipage}
\end{flushleft}


\end{multicols}


% СТРАНИЦА С ТАБЛИЦАМИ
% СТРАНИЦА С ТАБЛИЦАМИ
\newpage
\thispagestyle{empty}

\small

% ТАБЛИЦЫ 1 и 2 (
\begin{center}
% Таблица 1
\begin{minipage}[t]{0.45\textwidth}
\centering
\renewcommand{\arraystretch}{1.2}
\begin{tabular}{|c|c|}
\hline
$0\ldots 0$ & $l\ldots l$ \\
$1\ldots 1$ & $l-1\ldots l-1$ \\
$\vdots$ & $\vdots$ \\
$l-1\ldots l-1$ & $1\ldots 1$ \\
$l\ldots l$ & $0\ldots 0$ \\
\hline
\end{tabular}
\par\vspace{2mm}
$\underbrace{\hspace{2.4cm}}_{r\text{ столбцов}}$
$\underbrace{\hspace{2.4cm}}_{r\text{ столбцов}}$
\par\vspace{2mm}
\textbf{Таблица 1.}
\end{minipage}
\hfill
% Таблица 2
\begin{minipage}[t]{0.45\textwidth}
\centering
\renewcommand{\arraystretch}{1.2}
\begin{tabular}{|c|c|c|}
\hline
$0\ldots 0$ & $l\ldots l$ & $l+q$ \\
$1\ldots 1$ & $l-1\ldots l-1$ & $l+q-1$ \\
$\vdots$ & $\vdots$ & $\vdots$ \\
$l-1\ldots l-1$ & $1\ldots 1$ & $q+1$ \\
$l\ldots l$ & $0\ldots 0$ & $q$ \\
\hline
\end{tabular}
\par\vspace{2mm}
% ПЕРВАЯ скобка (под "0...0")
\makebox[0.2cm][c]{$\underbrace{\hspace{2.2cm}}_{r\text{ столбцов}}$}
\hspace{2mm}  % отступ между скобками
% ВТОРАЯ скобка (под "l...l") 
\makebox[3.6cm][c]{$\underbrace{\hspace{2.2cm}}_{(r-1)\text{ столбец}}$}
\par\vspace{2mm}
\textbf{Таблица 2.}
\end{minipage}
\end{center}

% ТАБЛИЦЫ 3, 4, 5 
\begin{center}
% Таблица 3
\begin{minipage}[t]{0.32\textwidth}
\centering
\renewcommand{\arraystretch}{1.2}
\begin{tabular}{|c|c|c|}
\hline
$0$ & $l$ & $2l$ \\
$2$ & $l-1$ & $2l-1$ \\
$4$ & $l-2$ & $2l-2$ \\
$\vdots$ & $\vdots$ & $\vdots$ \\
$2l$ & $0$ & $l$ \\
\hline
$1$ & $2l$ & $l-1$ \\
$3$ & $2l-1$ & $l-2$ \\
$5$ & $2l-2$ & $l-3$ \\
$\vdots$ & $\vdots$ & $\vdots$ \\
$2l-1$ & $l+1$ & $0$ \\
\hline
\end{tabular}
\par\vspace{2mm}
\textbf{Таблица 3.}
\end{minipage}
\hfill
% Таблица 4
\begin{minipage}[t]{0.32\textwidth}
\centering
\renewcommand{\arraystretch}{1.2}
\begin{tabular}{|c|c|c|}
\hline
$0$ & $l$ & $2l+1$ \\
$2$ & $l-1$ & $2l$ \\
$4$ & $l-2$ & $2l-1$ \\
$\vdots$ & $\vdots$ & $\vdots$ \\
$2l$ & $0$ & $l+1$ \\
\hline
$1$ & $2l$ & $l$ \\
$3$ & $2l-1$ & $l-1$ \\
$5$ & $2l-2$ & $l-2$ \\
$\vdots$ & $\vdots$ & $\vdots$ \\
$2l-1$ & $l+1$ & $1$ \\
\hline
\end{tabular}
\par\vspace{2mm}
\textbf{Таблица 4.}
\end{minipage}
\hfill
% Таблица 5
\begin{minipage}[t]{0.32\textwidth}
\centering
\renewcommand{\arraystretch}{1.2}
\begin{tabular}{|c|c|c|}
\hline
$0$ & $l-1$ & $2l$ \\
$2$ & $l-2$ & $2l-1$ \\
$4$ & $l-3$ & $2l-2$ \\
$\vdots$ & $\vdots$ & $\vdots$ \\
$2l-2$ & $0$ & $l+1$ \\
\hline
$1$ & $2l-1$ & $l-1$ \\
$3$ & $2l-2$ & $l-2$ \\
$5$ & $2l-3$ & $l-3$ \\
$\vdots$ & $\vdots$ & $\vdots$ \\
$2l-1$ & $l$ & $0$ \\
\hline
\end{tabular}
\par\vspace{2mm}
\textbf{Таблица 5.}
\end{minipage}
\end{center}

\vspace{10mm}

% Примечание
\raggedright
\textbf{Примечание:} $r$ столбцов означает, что соответствующий блок повторяется $r$ раз.
\end{document}